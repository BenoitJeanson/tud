\documentclass[11pt]{article}

\usepackage{sectsty}
\usepackage{graphicx}

\usepackage{svg}
\usepackage[off]{svg-extract}
\svgsetup{clean=true}
\usepackage{relsize}
\usepackage{subcaption}
\usepackage{wrapfig}
\usepackage{amsmath}
\usepackage{amsfonts} 
\usepackage{bm}
\usepackage{listingsutf8}
\usepackage{placeins}


\svgpath{{images}}

\usepackage{hyperref}
\usepackage{listings}
\usepackage{float}
\usepackage{booktabs}
\usepackage{makecell}

\usepackage{listings}
\usepackage{xcolor}

\setcounter{MaxMatrixCols}{20}

\definecolor{codegreen}{rgb}{0,0.6,0}
\definecolor{codegray}{rgb}{0.5,0.5,0.5}
\definecolor{codepurple}{rgb}{0.58,0,0.82}
\definecolor{backcolour}{rgb}{0.95,0.95,0.92}

\lstdefinestyle{mystyle}{
    backgroundcolor=\color{backcolour},   
    commentstyle=\color{codegreen},
    keywordstyle=\color{magenta},
    numberstyle=\tiny\color{codegray},
    stringstyle=\color{codepurple},
    basicstyle=\ttfamily\footnotesize,
    breakatwhitespace=false,         
    breaklines=true,                 
    captionpos=b,                    
    keepspaces=true,                 
    numbers=left,                    
    numbersep=5pt,                  
    showspaces=false,                
    showstringspaces=false,
    showtabs=false,                  
    tabsize=2
}
\lstset{style=mystyle}


% Margins
\topmargin=-0.45in
\evensidemargin=0in
\oddsidemargin=0in
\textwidth=6.5in
\textheight=9.0in
\headsep=0.25in

\def\rf{\emph{Risk Factor }}

\title{Project ET4350 Applied Convex Optimization

Change Detection in Time Series Model}
\author{Benoît Jeanson \\ Employee 930283 \\ \href{mailto:b.jeanson@tudelft.nl}{b.jeanson@tudelft.nl}}

\date{\today}

\begin{document}
\maketitle
\pagebreak

% Optional TOC
% \tableofcontents
% \pagebreak

%--Paper--

\section{Summary}
    This project consists in applying optimization methods to detect the time of a change of parameters of an autoregressive time series. After a short reminder of the problem, a first warm-up approach is developed to get familiar with the problem, then a formulation as an optimization problem is proposed. It is then adapted to be solvable by an off-the-shelf solver and solved, and then a simple subgradient method is implemented to solve it directly from its initial formulation.
    
    The project was developed in \emph{Python} -- as agreed with professor Geert Leus -- and run on a \emph{M1-chip MacBook Air}.

\section{Problem to solve}
    A scalar autoregressive (AR) time series is modeled as follows:
    $$y(t+3) = a(t)y(t+2) + b(t)y(t+1) + c(t)y(t) + c(t)y(t) + v(t); \quad v(t) \sim \mathcal{N}(0,0.5^2)$$
    with the assumption that the coefficient $a(t), b(t), c(t)$ are piecewise constant that change infrequently.

    The problem given in this assignment consists in identifying the parameters of the piecewise functions of the coefficients. As mentioned in the context of the assignment description, the problem concerns the detection of \emph{time(s)} of the change(s).

    The provided dataset includes $T=300$ samples of $y(t)$, as well as the coefficients that were used to generate it. The parameters only change once in the sequence. See Figure\ref{fig:dataset}.

    \begin{figure}[h]
        \centering
        \includesvg[scale=0.75, pretex=\tiny]{images/dataset.svg}
        \caption{\label{fig:dataset} The top graphic represents the autoregressive curve, and the bottom one the parameters used to generate it.}
    \end{figure}

    To lighten the notations, we will use the following notation for any variable $x$: $x_t$ instead of $x(t)$, and therefore any time series $x$ becomes naturally a vector.

\section{Simple regression}
    \label{section:regression}
    First, we start our approach by working on a part of the dataset for which the parameters are constant. The problem is reduced to a simple regression problem with $a, b, c$ as parameters. For the given time series, we know that this is the case for the interval [0, 49] that we will consider in this section.

    In this section let $N=49$, $Y_k = (y_k, y_{k+1} \cdots y_{k+N-3})^T \quad k \in \{0,1,2,3\}$, let $Z=[Y_2, Y_1, Y_0]$ and $X=(a, b, c)^T$.
    
    The regression problem can therefore be formulated as follows:
    $$ \min_{X}{\lVert Y_3 - ZX \rVert} $$

    Considering the fact that, the noise in the dataset is Gaussian, the Euclidean norm is the most appropriate. The problem is equivalent to (i.e. not taking the square root of the norm, which is a strictly increasing function):           
     \begin{equation*}
        \begin{aligned}
            & \min_{a,b,c}{f} \\
            \text{where} \quad &f(a,b,c) =  \sum_{i\in[0, N-3]}{(y_{i+3}  - a y_{i+2} - b  y_{i+1} - c y_i)^2}
        \end{aligned}
     \end{equation*}
                
    This first modeling was checked to ensure its efficiency and the results are displayed in figure \ref{fig:regression}. The identified parameters $(a, b, c) = (1.168, -0.988, 0.17)$ differs quite significantly from the original parameters $(1, -0.8, -0.2)$. However, the identified parameters better reflect the samples than the original ones for that small set of noisy samples. This was confirmed by the calculation of squared error of each extrapolated curves compare to the original values: i.e. $f(a,b,c)$ ends up to $10.52$ for the regressed parameters and $11.08$ for the original ones.

    \begin{figure}[h]
        \centering
        \includesvg[scale=0.75, pretex=\tiny]{images/regression.svg}
        \caption{\label{fig:regression} The \emph{original} curve represents the original samples $y_{t+3}$. The two other curves display a $\tilde{y_t} = ay_{t+2} + by_{t+1}+cy_t$. They are based on the 3 preceding values of $y$ time series to extrapolate what should be the next value if there was no noise. The \emph{original\_parameters} curve takes the parameters that were used to generate the initial samples set, the \emph{regressed\_parameters} one takes the parameters resulting from the regression.}
    \end{figure}
    
    This first analysis gives us confidence in the ability to identify the parameters of the autoregressive function: the regression optimization gives satisfactory results that enables to have good extrapolation features, but in the meantime, we see that we should not have high expectation on the precision on the parameters.

\section{Modeling of the problem}
    \subsection{Time detection as a cardinality problem}
        The time detection algorithm's goal consists in detecting the time at which the parameters may have changed - considering this should not happen often. This can be formulated as a cardinality problem on the variation of the parameters. We would define the maximum number of times the parameters could change.
        In this subsection, we work on the full data set: $N = T-1 = 299$. The parameters now includes the time: $A=[a_0, \ldots a_{N-3}]$ and similarly for $B$ and $C$.
        
        To extend the previous regression lost function, we need to redefine :
        \begin{equation*}
            \begin{aligned}
                X &= [A, B, C] \\
                Z &= (diag(Y_2), diag(Y_1), diag(Y_0))\\
                &= \begin{pmatrix} 
                    y_2 & 0 & \ldots & 0 & y_1 & 0 & \ldots & 0 & y_0 & 0 & \ldots \\
                    0 & y_3 & 0 & \ldots & 0 & y_2 & 0 & \ldots & 0 & y_1 & 0 & \ldots \\
                    0 & 0 & \ddots & 0 & \ldots & 0 & \ddots & 0 & \ldots & 0 & \ddots & 0 & \\
                    0 & \cdots & 0 & y_{N} & 0 & \ldots & 0 & y_{N-1} & 0 & \ldots & 0 & y_{N-2}
                \end{pmatrix} \\
                f (X) &= \lVert ZX - Y_3 \rVert^2_2
            \end{aligned}
        \end{equation*}

        $Y_3$ remaining as defined in section \ref{section:regression}. 
        
        Let $D_A = (a_{i+1} - a_i)_{i \in \{0,\ldots 3*(N-4)\}}$, as well as $D_B$, and $D_C$. These vectors reflect the variation in the parameters that we intend to limit the number of.

        Let $\mathcal{D} = \{0,1,\ldots 3*(N-4)\}\setminus\{N-4, 2*(N-4), 3*(N-4)\} $
        so that $ D_X = (D_A, D_B, D_C) = (x_{i+1} - x_i)_{i \in \mathcal{D}} $ the vector that defines the variation of the parameters. 
        
        Thus, the cardinality problem to solve becomes:
        \begin{equation*}
            \begin{aligned}
                \min_{X}&{\lVert ZX - Y_3 \rVert^2_2} \\
                    s.t.    & \bm{Card}(D_X) \le M
            \end{aligned}
        \end{equation*}

        $M$ being the maximum number of variations of the parameters.
    
    \subsection{Turn the formulation into a relaxed convex problem to solve it}
        As formulated in the previous subsection, the problem is very difficult to solve. The cardinality constraint is not convex and prevents us to use all the convenient tooling. We will then approach the problem by relaxing the cardinality constraint by adding a parameters' variation absolute value term in the cost function. By doing so, the solver will reduce as much as possible the variations. As, unlike the square function, the  gradient of the absolute value does not tend to zero, the solver will be incentivized to drag the content down to zero.

        The problem to solve becomes:
        \begin{equation}
            \label{eq:equation abs}
                \min_{X}{\sum_{i\in[0, N-3]}{(y_{i+3}  - a_i y_{i+2} - b_i y_{i+1} - c_i y_i)^2} + \mu\sum_{i \in \mathcal{D}}{|x_{i+1} - x_i|}}
        \end{equation}
        
        where $\mu$ is to be chosen to balance the cost function between the regression part and the limitation of the variations of the parameters. The higher $\mu$, the more the variation will be limited to the detriment of the quality of the regression.

        Formulated as is, there is no reason for the result to end up to a perfect solution. However, based on the result, we could have an identification of the timestamp at which a higher change is required to the parameters. That is a time detection. Then in a second step, a regression problem can be achieved limiting the variables of the problem to the constant of the piecewise function knowing the time of their respective disruptions.

\section{Solving with an off-the-shelf solver (\emph{Ipopt})}
    \subsection{Modeling: dealing with the absolute value}
    The chosen off-the-shelf solver is \emph{Ipopt} using \emph{pyomo} for the modeling. It is an interior point solver.
    
    First, it raised an error when submitting a model with an absolute value. We then had to use a known trick to transform an absolute value into a linear program.

    To get $|x|$, we add a linear constraint including two new variables $x = t^+ - t^- \quad t^+,t^- \in \mathbb{R}_+$ and then replacing $|x|$ in the objective function by $t^+ + t^-$. As the objective function is to be minimized, one of both will be set to its lower limit which is zero, and the other one will be equal to the absolute value.
    
    Thus, the formulation of the problem for \emph{Ipopt} becomes:
    \begin{equation*}
        \begin{aligned}
            \min_{X}&{\lVert ZX - Y_3 \rVert^2_2 + \mu\sum_{i \in \mathcal{D}}{t^+_i + t^-_i}} \\
            s.t. \quad &x_{i + 1} - x_{i} + t_i^+ - t_i^- = 0 \quad &\forall i \in \mathcal{D} \\
            &t_i^+, t_i^- \in \mathbb{R}_+ &\forall i \in \mathcal{D}
        \end{aligned}
    \end{equation*}

    \subsection{Results}
    The results are shown on figure \ref{fig:ipopt_time_detection}. After some trials, the parameter $mu=10$ gave the accurate results. The identification of the parameters $a(t), b(t), c(t)$ are very similar to the one shown on figure \ref{fig:dataset} even if not perfect. However, if we consider that what was investigated with this algorithm is the time of the change, we can observe that the max deviation of each parameter exactly point to the time of the change in the original time series.

    \begin{figure}[h]
        \centering
        \includesvg[scale=0.75, pretex=\tiny]{images/ipopt_time_detection.svg}
        \caption{\label{fig:ipopt_time_detection} The first graphic shows the curve of the deviations of the parameters. It allows the detection of the times where there is a significant change in the parameters. The second chart presents the values of the parameters as they are identified by the regression, and the last one shows the extrapolated values with the identified \emph{moedeled} parameters compare to the initial time series.}
    \end{figure}

    In a second step, we took the information of the times of change and solved the regression problem including the time of change for each parameter. The results are shown in the figure \ref{fig:final regression} where we can see that the mean of the difference between the extrapolated signals (regressed or original) are both very closed to the expected one (the noise is $\mathcal{N}(0, 0.5^2)$)

    \begin{figure}[h]
        \centering
        \includesvg[scale=0.5, pretex=\tiny]{images/final_regression.svg}
        \caption{\label{fig:final regression} Obtained with \emph{Ipopt}. The first column shows the result of the final regression for which the times of the change were given as constraint, the second column shows the original data. The first line shows the evolution of the parameters, the second, the evolution of the time series extrapolated with the corresponding parameters, and the original time series. The third line shows the difference between the two previous curves, and the last line shows the distribution of the identified noise.}
    \end{figure}

    \subsection{Performances}
        We focus only on the time detection, which is the core of the project. The log of the solver state the number of variables is $2667$, the number of equality constraints is $888$.

        The solver found an \emph{optimal solution} in $15$ iterations that took in total $0.034$ seconds.

    \FloatBarrier

\section{Low complexity-algorithm implementation}
    The problem raised by the absolute value was appealing. That was an opportunity to implement a sub-gradient method algorithm to deal with the \emph{non-differentiable} character of the absolute value function. The developed algorithm is based on equation \ref{eq:equation abs} which is a convex non-constrained optimization problem.
    
    The code of the algorithm is given in appendix \ref{appendix: subgradient method}.

    The implementation follows the method given during the lecture. A lot of degrees of freedom are given. Especially various options where possible to set up the step length $\alpha$ progression : constant step size, constant step length, square summable but no summable, nonsummable diminishing, nonsummable diminishing step lengths.
    All were tested, for all of them the convergence was very slow.
    
    Finnally we ended up with square summable but not summable function $\alpha_k = a/(b + k)$ with $a = 1$ and $b = 200$. And the parameter $\mu=10$.

    No less than $1000$ iterations -- performed in 1 second -- are needed to be able to get a clear identification of the index of the parameters change (see figure \ref{fig:subgradient1000}), but the parameters $a,b$ and $c$ are far from being identifiable as piecewise constant functions.

    \begin{figure}[h]
        \centering
        \includesvg[scale=0.75, pretex=\tiny]{images/subgradient1000.svg}
        \caption{\label{fig:subgradient1000} Obtained with the subgradient method in $1,000$ iterations. The first graphic shows the time series of the parameters $a,b,c$. And the second shows their variations along the samples. The time of the change of the parameters are clearly identifiable in the second graphic, even if the parameters $a,b$ and $c$ are still far from being identifiable as piecewise constant functions.}
    \end{figure}

    Figure \ref{fig:subgradient} shows the results after $100,000$ iteration in $145$ seconds. The time series of the parameters $a,b$ and $c$ are tending towards the expected shape. But it is at a cost of a lot of iterations, and the results with the \emph{Ipopt} were more accurate.

    \begin{figure}[h]
        \centering
        \includesvg[scale=0.75, pretex=\tiny]{images/subgradient.svg}
        \caption{\label{fig:subgradient} Obtained with the subgradient method in $100,000$ iterations. The first row shows the objective function over the iterations. The first one is the total result, and the two following show the squared error -- which correspond to the regression part -- and the cost of the deviation of the parameters. The beginning starts with a huge error (the subgradient method is not a descent method). The second row present first the time series of the parameters $a,b$ and $c$, their deviation, and the evolution of one difference between two value of $a$ against the iterations. The parameters time series are tending towards piecewise constant function.}
    \end{figure}


    \FloatBarrier

    Some attempts to improve the convergence were unsuccessful on the non-differentiable term of the objective function (the absolution value). For example for a low value of $x_{i+1} - x_{i}$, giving a null value to the subgradient to try to stabilize the evolution of the objective function considering this is one possible subgradient. This did not help, but did not deteriorate. One strongest attempt was to force neighboring values of $x$ to be equal when their difference was low. This dramatically deteriorated the results. Indeed, this last approach has no strong mathematical support... but should be tested.

    One other attempt was inspired by the interior point method, and consisted in approximating the absolute value with $log(cosh \lambda x)/\lambda$, which did not give substantial improvements, and introduced computation time.

\section{Conclusion}
This assignment was an opportunity to dig into the practical aspects of \emph{convex optimization} and to have a glimpse on that powerful approach to problem-solving. Looking at the project descriptions at the beginning of the course, it was puzzling as at first sight it was not an obvious optimization problem.

The formulation as a cardinality problem was a call to approximating formulation to enter in to the field of convex optimization.

The results given by \emph{Ipopt} demonstrates that the theory -- especially around the duality -- allows stunning efficiency. Only 15 iterations to reach optimality!

Therefore, humility is the concluding feeling. Indeed, the implemented subgradient method -- which is a first order one -- enabled to identify the time of the change, but the final curves remained suboptimal and they were obtained at the cost of $100,000$ iterations.

\pagebreak

\appendix
The appendixes take the key snippets corresponding to the assignment. The complete \emph{Jupyter notebook} is delivered with the present report.
\section{Implementation with \emph{pyomo} and \emph{Ipopt}}

    \subsection{Identification of the change detection}
    \label{appendix: pyomo and Ipopt}
        \begin{lstlisting}[language=Python]
            def change_detection(cd_y, sample_size, mu):
            '''
            Build a pyomo.AbstractModel for the change detection of the parameters.
            the model build stores the y time sereis in model.y, and the time series of the parameters a, b, and c in a 3 column time series model.w.
            model.d_abs takes the t+ (index:1) and t- (index:2) values for the modeling of the absolute value of the variation of the parameters.
            
            Parameters
                cd_y : the time series y(t)
                sample_size : the size of the samples to be considered
                mu: the parameter mu that gives a weight to the deviation cost.

            Returns
                model: the model once resolved
                report: the report from the solver    
            '''
            model = pyo.AbstractModel()
            model.data_index_set = pyo.RangeSet(0, sample_size - 1)
            model.y = pyo.Param(model.data_index_set, initialize = lambda _, i: cd_y[i])

            model.var_index_set = model.data_index_set - {sample_size - 1, sample_size - 2, sample_size - 3}
            model.var_dim_set = pyo.RangeSet(0, 2)
            
            model.w = pyo.Var(model.var_dim_set, model.var_index_set)

            model.var_d_set = model.var_index_set - {sample_size - 4}
            model.d_abs = pyo.Var(model.var_dim_set, model.var_d_set, {1 , 2}, domain = pyo.NonNegativeReals)
            
            model.contr_d = pyo.Constraint(model.var_dim_set, model.var_d_set,
                                        rule = lambda m, i, j: m.d_abs[i, j, 2] - m.d_abs[i, j, 1]  == m.w[i, j + 1] - m.w[i, j])

            def rec_expr(m, i):
                return  m.y[i + 3] \
                        - m.w[0, i] * m.y[i + 2] \
                        - m.w[1, i] * m.y[i + 1] \
                        - m.w[2, i] * m.y[i]

            def obj_expr(m) :
                return sum(rec_expr(m, i)**2 for i in m.var_index_set) + mu * pyo.summation(m.d_abs)

            model.obj = pyo.Objective(rule = obj_expr)

            model.construct()

            solver = pyo.SolverFactory("ipopt")
            report = solver.solve(model, logfile = 'log.txt')
            return model, report       
        \end{lstlisting}

    \subsection{Final regression}
    \begin{lstlisting}[language=Python]
        def identify_change_index(model):
        '''
        Identify the index at which the most probable change in the parameters are observed. For each of the 3 parameters a, b, c, the most probable change is the identified as the max deviation index.
        
        Parameter
            model: the model generated by the change_detection function
        
        Return
            a tuple with the 3 indexes corresponding to the 3 parameters.
        '''
        res = []
        for dim in {0, 1, 2}:
            list = [model.d_abs[dim, i, 2].value + model.d_abs[dim, i, 1].value for i in model.var_d_set]
            res.append(list.index(max(list)))
        return tuple(res)
    
    def final_regression(cd_y, sample_size: int, change_indexes: tuple[int, int, int]):
        '''
        Performs a regression to identify two values for each parameters a,b and c, considering each of them have one change at one respective index.
        
        Parameter
            cd_y: the time series y(t)
            sample_size: the size of the samples to be considered
            change_indexes: a tuple with the 3 indexes corrsesponding to each of the 3 parameters. This could result from the function identify_change_index
        
        Returns
            model: the pyomo model solved
            report: the report from the solver
        '''
        model = pyo.AbstractModel()
        model.data_index_set = pyo.RangeSet(0, sample_size - 1)
        model.y = pyo.Param(model.data_index_set, initialize = lambda _, i: cd_y[i])
    
        model.var_index_set = model.data_index_set - {sample_size - 1, sample_size - 2, sample_size - 3}
        model.var_dim_set = pyo.RangeSet(0, 2)
        
        model.w = pyo.Var(model.var_dim_set, {0, 1})
    
        def rec_expr(m, i):
            return  m.y[i + 3] \
                    - (m.w[0, 0] if i<=change_indexes[0] else m.w[0,1]) * m.y[i + 2] \
                    - (m.w[1, 0] if i<=change_indexes[1] else m.w[1,1]) * m.y[i + 1] \
                    - (m.w[2, 0] if i<=change_indexes[2] else m.w[2,1]) * m.y[i]
    
        def obj_expr(m) :
            return sum(rec_expr(m, i)**2 for i in m.var_index_set)
    
        model.obj = pyo.Objective(rule = obj_expr)
    
        model.construct()
    
        solver = pyo.SolverFactory("ipopt")
        report = solver.solve(model, logfile = 'log.txt')
        return model, report
    \end{lstlisting}
    

\section{Implementation of the subgradient method}
\label{appendix: subgradient method}
    \subsection{Subgradient function}
        \begin{lstlisting}[language=Python]
                def subgradient_method(fun_and_grad, x_0, alpha, min_k, max_k):
            '''
            The subgradient method implementation.
            
            Parameters
                funa_and_grad: a function that takes a vector x as input, and returns the value of the function f(x), and a vector corresponding the the gradient df(x)
                x_0: the starting point for x
                alpha: a function that takes the iteration numbber, and the gradient, and returns the step size alpha
                min_k: the minimum iteration at which the results are considered
                max_k: the maximum number of iteration
            
            Returns
                f_min: the minimum value reached for the objective function f
                x_min: the value of x that corresponds to f_min
                k_min: the iteration at which f-min was reached
                f_k: a time series of the value of f against the iterations
                log_time_series: time series of values given by fun_and_grad that are stored as log   
            '''
            f_x, _, _ = fun_and_grad(x_0)
            f_k = [f_x]
            f_min = f_x
            k_min = 0
            x = x_0
            x_min = np.copy(x_0)
            k = 0
            log_time_series = []
            while(k <= max_k):
                f_x, d_f_x, log = fun_and_grad(x)
                if f_x<f_min or k == min_k:
                    f_min = f_x
                    k_min = k
                    x_min = np.copy(x)
                log_time_series.append(log)
                delta_x = np.multiply(d_f_x, -alpha(k, d_f_x))
                x = np.add(x, delta_x)
                f_k.append(f_x)
                k += 1
            return f_min, x_min, k_min, f_k, log_time_series
        \end{lstlisting}

    \subsection{Various implementations of the function \emph{alpha}}
        \begin{lstlisting}[language=Python]
            alpha_const = lambda alpha: (lambda k, df: alpha)
            alpha_constant_step_length = lambda gamma: (lambda k, df: gamma/np.linalg.norm(df))
            alpha_sq_summable = lambda a, b: (lambda k, df: a / (b+k))
            alpha_nonsummable_diminishing = lambda a: lambda k,df : a/np.sqrt(k+1)
            alpha_nonsummable_diminishing_step_length = lambda gamma : lambda k, df : gamma/np.linalg.norm(df)/np.sqrt(k+1)
        \end{lstlisting}

    
    \subsection{The fun\_and\_grad function for the time detection}
    The subgradient method is called as follows:
        \begin{lstlisting}[language=Python]
            fun_and_grad = lambda y, sample_size, mu: lambda x: objective_and_grad(y, x, sample_size, mu)
            f_min, x_min, k_min, f_k, log = subgradient_method(fun_and_grad(cd_y, sample_size, mu), x0, alpha, min_k, max_k)
        \end{lstlisting}
    
    With the implementations:

    \begin{lstlisting}[language=Python]
        def objective_and_grad(y, x, sample_size, mu):
        '''
        Returns the value of the objective function f(x) for the time detction and a vector corresponding to a subgradient df(x).
        
        Parameters
            y: the time series y(t)
            x: the value of the parameters a,b,c flattened into one single vector
            sample_size the number of samples to be considered
            mu: the weight given to the variation of the parameters in the objective function
        
        Returns
            res: the value of f(x)
            res_subgrads: a vector corresponding to the subgradients of f(x)
            some log to monitor the [the values returned by sqr_error, and d_abs_error,  the value of x[11]-x[10]]
        '''
        res = 0
        subgrads = []
        err_sum, d_err_sum = 0,0
        for i in range(sample_size - 4):
            f_err, subgrads_err = sqr_error(y, x, sample_size, i)
            f_d_error, subgrads_d_err = d_abs_error(x, sample_size, i, mu)
            res += f_err + f_d_error
            err_sum += f_err
            d_err_sum += f_d_error
            subgrads += subgrads_err
            subgrads += subgrads_d_err
        
        i = sample_size - 4
        f_err, subgrads_err = sqr_error(y, x, sample_size, i)
        res += f_err
        err_sum += f_err
        subgrads += subgrads_err
    
        res_subgrads = np.zeros((sample_size - 3) * 3)
        for sg in subgrads:
            res_subgrads[sg[0]] += sg[1]
        
        return res, res_subgrads, [err_sum, d_err_sum, x[11] - x[10]]
        
        def sqr_error(y, x, sample_size, i):
    '''
    Calculates the squared error and gradient of the i^th index of the parameters a, b, and c of the autoregressive function with regards to the y time series.
    
    Parameters
        y: the time series y(t)
        x: the time series of the parameters a,b,c flattened into one single vector x
        sample_size: the number of samples to be considered
        i: the considered index
    
    Returns
        the squared error
        the contributions of the i^th index to the gradient as a list of tuples. Each tuple is shaped as (index in the vector x, value of the gradient)
    '''
    size = sample_size - 3
    f = y[i + 3] \
            - x[i] * y[i + 2]\
            - x[size + i] * y[i + 1]\
            - x[size * 2 + i] * y[i]
    subgrads_f = [(i, -2 * y[i + 2] * f),
                  (size + i, -2 * y[i + 1] * f),
                  (size * 2 + i, -2 * y[i] * f)]
    return f*f, subgrads_f

    def d_abs_error(x, sample_size, i, mu):
    '''
    Calculates the absolute error and subgradient of the i^th index of the variation of the parameters a, b, and c.
    
    Parameters
        x: the time series of the parameters a,b,c flattened into one single vector x
        sample_size: the number of samples to be considered
        i: the considered index
        mu: the weight given to the variation of the parameters in the objective function
    
    Returns
        the squared error
        the contributions of the i^th index to the gradient as a list of tuples. Each tuple is shaped as (index in the vector x, value of the gradient)
    '''
    size = sample_size - 3
    f = 0
    subgrads_f = []
    for k in range(3):
        delta_x = x[k * size + i + 1] - x[k * size + i]
        a = mu * (delta_x)
        f += abs(a)
        subgrads_f.append((k * size + i + 1, mu if a > 0 else -mu))
        subgrads_f.append((k * size + i,    -mu if a > 0 else  mu))
    return f, subgrads_f

    \end{lstlisting}


\end{document}